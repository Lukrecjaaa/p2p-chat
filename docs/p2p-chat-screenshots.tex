\documentclass[a4paper,12pt]{article}

\usepackage{newtxtext,newtxmath}

\usepackage[T1]{fontenc}
\usepackage[utf8]{inputenc}
\usepackage[polish]{babel}

\usepackage{geometry}
\geometry{left=3cm, right=3cm, top=3cm, bottom=3cm, includeheadfoot}

\renewcommand{\baselinestretch}{1.5}
\setlength{\parindent}{1.25cm}
\setlength{\parskip}{0pt}
\raggedbottom
\emergencystretch=2em

\usepackage{graphicx}
\usepackage{float}
\usepackage{caption}

\begin{document}

\begin{center}
    \textbf{Załącznik 2. Zrzuty ekranu aplikacji \texttt{p2p-chat}}
\end{center}

\section*{1. Interfejs tekstowy (TUI)}

Na rysunku~\ref{fig:tui} przedstawiono przykładowy widok interfejsu tekstowego aplikacji \texttt{p2p-chat}. Zrzut ekranu obejmuje jednoczesną pracę trzech węzłów uruchomionych w jednym terminalu:
\begin{itemize}
    \item po lewej stronie widoczny jest węzeł użytkownika pracujący w trybie podglądu logów, prezentujący zdarzenia sieciowe oraz komunikaty diagnostyczne;
    \item w środkowym oknie pokazano tryb czatu (\texttt{ChatMode}) dla węzła klienta, obejmujący listę wiadomości oraz pole wprowadzania nowych komunikatów;
    \item po prawej stronie widoczny jest węzeł pełniący rolę \textit{mailboxa}, odpowiedzialny za przechowywanie wiadomości dla nieobecnych uczestników.
\end{itemize}

Widok ten ilustruje sposób pracy systemu w środowisku terminalowym, w którym użytkownik ma równoczesny dostęp do historii rozmów, bieżących logów oraz informacji o pracy węzła \textit{mailbox}.

\begin{figure}[H]
    \centering
    \includegraphics[width=0.95\textwidth]{tui-screenshot.png}
    \caption{Interfejs tekstowy (TUI) z trzema węzłami: użytkownik w trybie logów, klient w trybie czatu oraz węzeł mailbox.}
    \label{fig:tui}
\end{figure}

\section*{2. Interfejs webowy}

Rysunek~\ref{fig:webui} przedstawia widok interfejsu webowego komunikatora \texttt{p2p-chat}. Na zrzucie ekranu widoczna jest aktywna konwersacja pomiędzy dwoma użytkownikami, prezentowana w oknie czatu wraz z przypisanymi kolorami oraz awatarami dopasowanymi do identyfikatorów \textit{PeerId}. Kolorystyka oraz grafiki awatarów są deterministycznie wyprowadzane z identyfikatora węzła, co ułatwia wizualne rozróżnianie rozmówców.

W prawym obszarze interfejsu widoczne są dodatkowe okna:
\begin{itemize}
    \item \textbf{System Status} -- panel prezentujący bieżące informacje o stanie węzła, połączeniu z siecią P2P oraz działaniu modułów pomocniczych;
    \item \textbf{Add Friend} -- okno umożliwiające dodanie nowego znajomego na podstawie jego identyfikatora, w tym klucza publicznego używanego do szyfrowania end-to-end;
    \item \textbf{Your Info} -- panel wyświetlający dane lokalnego użytkownika, w szczególności identyfikator \textit{PeerId} oraz klucz publiczny wykorzystywany do nawiązywania bezpiecznych połączeń.
    \end{itemize}

Zrzut ekranu obrazuje docelowy sposób korzystania z aplikacji przez użytkownika końcowego -- komunikacja odbywa się w przeglądarce, natomiast logika sieciowa i kryptograficzna jest realizowana przez lokalny węzeł Rust uruchomiony w tle.

\begin{figure}[H]
    \centering
    \includegraphics[width=0.95\textwidth]{web-ui-screenshot.png}
    \caption{Interfejs webowy z aktywną konwersacją oraz otwartymi panelami System Status, Add Friend i Your Info.}
    \label{fig:webui}
\end{figure}

\end{document}
